\chapter{Arhitektura i dizajn sustava}

		Arhitektura je podijeljena na tri podsustava:
    \begin{packed_item}
        \item Mobilna aplikacija
        \item Poslužitelj
        \item Poslužitelj baze podataka
    \end{packed_item}
    
    Okviri i jezici koristimo izabrani su imajući na umu, između ostaloga, njihovu relativno dugotrajnu popularnost, što znači da osim male vjerojatnosti skore zastare postoji i veliki skup korisnika koji pružaju podršku, kao i problema koje su ti korisnici već riješili, a dostupni su na stranicama poput \url{stackoverflow.com}. Također su izvrsno dokumentirani. \\
    
	    \begin{figure}[H]
	        \begin{center}
            \includegraphics[width=1.0\linewidth]{slike/arh_skica.png}
            \caption{Skica arhitekture sustava}
            \label{fig:arh_skica}
            \end{center}
        \end{figure}
        
    \underbar{Aplikacija na mobilnom uređaju} napravljena je za sustav Android. Dio je frontend sloja, ali i u sklopu operativnog sustava komunicira s ugrađenom lokalnom bazom podataka (dakle nema zasebni poslužitelj). Preko protokola HTTP komunicira s poslužiteljem. Mobilna aplikacija napisana je u strojno podržanom (native) jeziku Androida - Java - koji je objektno orijentirani jezik što nam omogućuje podjelu u razrede koji su u paketima grupirani u međusobno povezane elemente, a pritom nam omogućuje i zadržavanje apstrakcije. Omogućuje nam i ponovnu iskoristivost jer bilo koji uređaj koji ima Java virtualni stroj može pokrenuti isti kôd. Android kao programski okvir omogućava veliku fleksibilnost (kroz razmještanje i imenovanje resursa prema dogovoru), poput automatske promjene jezika ovisno o korisničkim postavkama ili rezolucije slika ovisno o rezoluciji ekrana. \\
    \underbar{Kôd na poslužitelju} dio je backend sloja i napisan je u Pythonu 3 koji se, osim spomenutim prednostima objektno orijentirane paradigme, odlikuje i jednostavnošću kodiranja što značajno smanjuje vrijeme razvoja. Pogotovo uz backend okvir \textit{django} koji omogućava fleksibilnost ne koristeći "Convention over configuration" paradigmu, ali je i dalje jednostavan jer je pisan u Pythonu. Njegovo svojstvo "batteries included" smanjuje količinu posla razvojnom timu akcijama poput automatskog stvaranja sučelja za administratora kojemu se može pristupiti iz \underbar{web preglenika} korištenjem HTTP protokola te jednostavnog povezivanja s email poslužiteljem pomoću SMTP protokola. \\
    Koristimo stilističku varijaciju arhitekture zasnovane na događajima: MVC (\textit{Model, View, Controller}) obrazac. Kako mu i ime kaže, dijeli aplikaciju na model, pogled (View) i nadglednik (Controller). Time se omogućava odvajanje korisničkog sučelja (kroz pogled) od ostatka sustava. Za upravljanje zahtjevima korisnika služi nadglednik, a model opisuje strukturu i pravila vezane uz podatke. Podržava ga django. \\
  
	
		

		

				
		\section{Baze podataka}
		
		Za spremanje podataka koristimo, zbog njihove uvriježenosti i uglavnom velike sličnosti s našim poimanjem onoga što predstavljaju, relacijske baze podataka. One dijele informacije o objektima u atribute, a vrste objekata u tablice, tj. relacije. Takve baze podataka upravo postoje i optimirane su za učinkovito spremanje čitanje i izmjenu podatka, pa ih zato za to i koristimo. Konkretno, na poslužitelju smo se odlučili za PostgreSQL, a na Android uređaju imamo SQLite3. \\
		
		Slijede opisi tablica i njihovih veza. Imena atributa primarnog ključa deblje su otisnuta, a pozadinska boja ćelije u kojoj je ime stranog ključa promijenjena je u plavu.
			
			
		%--------------------------------------------------------------------------------
		

            \subsection{Opis tablica na uređaju}

                Tablica \underbar{omiljeni} sadrži šifre stavaka koje su označene kao omiljene. Tablica omiljeni je u \textit{Many-to-one} vezi s tablicom stavka (preko atributa sifStavka).
                \begin{longtabu} to \textwidth {|X[6, l]|X[6, l]|X[20, l]|}
                    
                    \hline \multicolumn{3}{|c|}{\textbf{omiljeni}}     \\[3pt] \hline
                    \endfirsthead
                    
                    \hline \multicolumn{3}{|c|}{\textbf{omiljeni}}     \\[3pt] \hline
                    \endhead
                    
                    \hline 
                    \endlastfoot

                    \cellcolor{LightBlue} \textbf{sifStavka} & INT & šifra stavke koja je označena kao omiljena \\ \hline
                    
                    
                    
                \end{longtabu}

                Tablica \underbar{popis} sadrži nazive popisa i njihove načine izračuna cijene. Tablica popis je u \textit{One-to-many} vezi s tablicom stavka (preko atributa sifPopis).
                \begin{longtabu} to \textwidth {|X[6, l]|X[6, l]|X[20, l]|}
                    
                    \hline \multicolumn{3}{|c|}{\textbf{popis}}     \\[3pt] \hline
                    \endfirsthead
                    
                    \hline \multicolumn{3}{|c|}{\textbf{popis}}     \\[3pt] \hline
                    \endhead
                    
                    \hline 
                    \endlastfoot

                    \textbf{sifPopis} & INT & šifra popisa \\ \hline
                    nazivPopis & NCHAR VARYING & naziv popisa \\ \hline
                    izrCijene & BIT & način izračuna cijene \\ \hline
                    sifTrgovina & INT & šifra trgovine vezane uz popis (ako je primjenjivo) \\ \hline
                    
                    
                    
                \end{longtabu}

                Tablica \underbar{stavka} sadrži stavke na popisu. Tablica stavka je u \textit{One-to-many} vezi s tablicom omiljeni (preko atributa sifStavka). Tablica stavka je u \textit{Many-to-one} vezi s tablicom popis (preko atributa sifPopis).
                \begin{longtabu} to \textwidth {|X[6, l]|X[6, l]|X[20, l]|}
                    
                    \hline \multicolumn{3}{|c|}{\textbf{stavka}}     \\[3pt] \hline
                    \endfirsthead
                    
                    \hline \multicolumn{3}{|c|}{\textbf{stavka}}     \\[3pt] \hline
                    \endhead
                    
                    \hline 
                    \endlastfoot

                    \textbf{sifStavka} & INT & šifra stavke \\ \hline
                    \cellcolor{LightBlue} sifPopis & INT & šifra popisa na kojem je stavka \\ \hline
                    barkod & DECIMAL & barkod artikla \\ \hline
                    cijena & DECIMAL & cijena artikla \\ \hline
                    filtarFunkcija & VARCHAR & sadrži sažeto napisane filtar funkcije koje se primjenjuju \\ \hline
                    uKosarici & BIT & oznaka je li stavka dodana u košaricu \\ \hline
                    kolicina & DECIMAL & količina artikla u intervalu [0,000 - 999,999] \\ \hline
                    naziv & NCHAR VARYING & naziv artikla \\ \hline
                    sifTrgovina & INT & šifra trgovine vezane uz cijenu \\ \hline
                    
                    
                    
                \end{longtabu}



            \subsection{Opis tablica na posluzitelju}

                Tablica \underbar{artikl} sadrži popis svih barkodova artikala. Tablica artikl je u \textit{One-to-many} vezi s tablicom opisArtikla (preko atributa barkod) i tablicom artiklUTrgovini (preko atributa barkod).
                \begin{longtabu} to \textwidth {|X[6, l]|X[6, l]|X[20, l]|}
                    
                    \hline \multicolumn{3}{|c|}{\textbf{artikl}}     \\[3pt] \hline
                    \endfirsthead
                    
                    \hline \multicolumn{3}{|c|}{\textbf{artikl}}     \\[3pt] \hline
                    \endhead
                    
                    \hline 
                    \endlastfoot

                    \textbf{barkod} & DECIMAL & barkod artikla \\ \hline
                    
                    
                    
                \end{longtabu}

                Tablica \underbar{artiklUTrgovini} sadrži popis artikala u trgovini. Tablica artiklUTrgovini je u \textit{Many-to-one} vezi s tablicom artikl (preko atributa barkod), tablicom trgovina (preko atributa sifTrgovina) i tablicom opisArtikla (preko atributa barkod i email).
                \begin{longtabu} to \textwidth {|X[6, l]|X[6, l]|X[20, l]|}
                    
                    \hline \multicolumn{3}{|c|}{\textbf{artiklUTrgovini}}     \\[3pt] \hline
                    \endfirsthead
                    
                    \hline \multicolumn{3}{|c|}{\textbf{artiklUTrgovini}}     \\[3pt] \hline
                    \endhead
                    
                    \hline 
                    \endlastfoot

                    \cellcolor{LightBlue} \textbf{barkod} & DECIMAL & barkod artikla \\ \hline
                    \cellcolor{LightBlue} \textbf{sifTrgovina} & CHAR & šifra trgovine \\ \hline
                    \cellcolor{LightBlue} email & VARCHAR & email vezan uz željeni opis \\ \hline
                    cijena & DECIMAL & cijena artikla u trgovini \\ \hline
                    popust & FLOAT & popust na cijenu artikla u trgovini \\ \hline
                    dostupnost & BIT & ima li artikla u trgovini \\ \hline
                    
                    
                    
                \end{longtabu}

                Tablica \underbar{kategorija} sadrži popis kategorija artikala. Tablica kategorija je u \textit{One-to-many} vezi s tablicom potkategorija (preko atributa sifKategorija).
                \begin{longtabu} to \textwidth {|X[6, l]|X[6, l]|X[20, l]|}
                    
                    \hline \multicolumn{3}{|c|}{\textbf{kategorija}}     \\[3pt] \hline
                    \endfirsthead
                    
                    \hline \multicolumn{3}{|c|}{\textbf{kategorija}}     \\[3pt] \hline
                    \endhead
                    
                    \hline 
                    \endlastfoot

                    \textbf{sifKategorija} & INT & šifra kategorije artikla \\ \hline
                    nazivKategorija & NCHAR VARYING & naziv kategorije artikla \\ \hline
                    
                    
                    
                \end{longtabu}

                Tablica \underbar{korisnik} sadrži informacije o prijavljenim korisnicima aplikacije. Tablica korisnik je u \textit{One-to-many} vezi s tablicom onemoguceniRacun (preko atributa email), tablicom opisArtikla (preko atributa email), tablicom privremenaLozinka (preko atributa email) i tablicom trgovina (preko atributa email). Tablica korisnik je u \textit{Many-to-one} vezi s tablicom uloga (preko atributa sifUloga).
                \begin{longtabu} to \textwidth {|X[6, l]|X[6, l]|X[20, l]|}
                    
                    \hline \multicolumn{3}{|c|}{\textbf{korisnik}}     \\[3pt] \hline
                    \endfirsthead
                    
                    \hline \multicolumn{3}{|c|}{\textbf{korisnik}}     \\[3pt] \hline
                    \endhead
                    
                    \hline 
                    \endlastfoot

                    \textbf{email} & VARCHAR & email korisnika \\ \hline
                    \cellcolor{LightBlue} sifUloga & INT & šifra uloge korisnika \\ \hline
                    lozinka & VARCHAR & hash lozinke korisnika (SHA 256) \\ \hline
                    token & CHAR & token sjednice korisnika \\ \hline
                    
                    
                    
                \end{longtabu}

                Tablica \underbar{onemoguceniRacun} sadrži popis onemogućenih računa. Tablica onemoguceniRacun je u \textit{Many-to-one} vezi s tablicom korisnik (preko atributa adminEmail).
                \begin{longtabu} to \textwidth {|X[6, l]|X[6, l]|X[20, l]|}
                    
                    \hline \multicolumn{3}{|c|}{\textbf{onemoguceniRacun}}     \\[3pt] \hline
                    \endfirsthead
                    
                    \hline \multicolumn{3}{|c|}{\textbf{onemoguceniRacun}}     \\[3pt] \hline
                    \endhead
                    
                    \hline 
                    \endlastfoot

                    \cellcolor{LightBlue} \textbf{onemoguceni} & VARCHAR & email osobe kojoj je onemogućen pristup \\ \hline
                    \cellcolor{LightBlue} \textbf{adminEmail} & VARCHAR & email admina koji je onemogućio osobi pristup \\ \hline
                    datum & DATE & datum onemogućenja \\ \hline
                    
                    
                    
                \end{longtabu}

                Tablica \underbar{opisArtikla} sadrži opis i informacije vezane uz artikl zadan barkodom koje je napisao neki prijavljeni korisnik. Tablica opisArtikla je u \textit{One-to-many} vezi s tablicom artiklUTrgovini (preko atributa barkod i email). Tablica opisArtikla je u \textit{Many-to-one} vezi s tablicom artikl (preko atributa barkod), tablicom korisnik (preko atributa email), tablicom vrsta (preko atributa sifKategorija, sifPotkategorija i sifVrsta) i tablicom zemlja (preko atributa sifZemlja).
                \begin{longtabu} to \textwidth {|X[6, l]|X[6, l]|X[20, l]|}
                    
                    \hline \multicolumn{3}{|c|}{\textbf{opisArtikla}}     \\[3pt] \hline
                    \endfirsthead
                    
                    \hline \multicolumn{3}{|c|}{\textbf{opisArtikla}}     \\[3pt] \hline
                    \endhead
                    
                    \hline 
                    \endlastfoot

                    \cellcolor{LightBlue} \textbf{barkod} & DECIMAL & barkod artikla \\ \hline
                    \cellcolor{LightBlue} \textbf{email} & VARCHAR & email osobe koja je napisala opis \\ \hline
                    \cellcolor{LightBlue} sifKategorija & INT & šifra kategorije artikla \\ \hline
                    \cellcolor{LightBlue} sifPotkategorija & INT & šifra potkategorije artikla \\ \hline
                    \cellcolor{LightBlue} sifVrsta & INT & šifra vrste artikla \\ \hline
                    \cellcolor{LightBlue} sifZemlja & CHAR & šifra zemlje podrijetla \\ \hline
                    kratkiOpis & NCHAR VARYING & kratki opis artikla (do 255 znakova) \\ \hline
                    nazivArtikla & NCHAR VARYING & naziv artikla (do 32 znaka) \\ \hline
                    brojGlasova & INT & broj glasova o opisu artikla (goreglasovi - doljeglasovi) \\ \hline
                    masa & INT & masa artikla \\ \hline
                    
                    
                    
                \end{longtabu}

                Tablica \underbar{potkategorija} sadrži popis potkategorija artikala. Tablica potkategorija je u \textit{One-to-many} vezi s tablicom vrsta (preko atributa sifKategorija i sifPotkategorija). Tablica potkategorija je u \textit{Many-to-one} vezi s tablicom kategorija (preko atributa sifKategorija).
                \begin{longtabu} to \textwidth {|X[6, l]|X[6, l]|X[20, l]|}
                    
                    \hline \multicolumn{3}{|c|}{\textbf{potkategorija}}     \\[3pt] \hline
                    \endfirsthead
                    
                    \hline \multicolumn{3}{|c|}{\textbf{potkategorija}}     \\[3pt] \hline
                    \endhead
                    
                    \hline 
                    \endlastfoot

                    \cellcolor{LightBlue} \textbf{sifKategorija} & INT & šifra kategorije kojoj potkategorija artikla pripada \\ \hline
                    \textbf{sifPotkategorija} & INT & šifra podkategorije artikla \\ \hline
                    nazivPotkategorija & NCHAR VARYING & naziv potkategorije artikla \\ \hline
                    
                    
                    
                \end{longtabu}

                Tablica \underbar{privremenaLozinka} sadrži privremene lozinke poslane na email korisnika. Tablica privremenaLozinka je u \textit{Many-to-one} vezi s tablicom korisnik (preko atributa email).
                \begin{longtabu} to \textwidth {|X[6, l]|X[6, l]|X[20, l]|}
                    
                    \hline \multicolumn{3}{|c|}{\textbf{privremenaLozinka}}     \\[3pt] \hline
                    \endfirsthead
                    
                    \hline \multicolumn{3}{|c|}{\textbf{privremenaLozinka}}     \\[3pt] \hline
                    \endhead
                    
                    \hline 
                    \endlastfoot

                    \cellcolor{LightBlue} \textbf{email} & VARCHAR & email korisnika koji je zatrazio promjenu \\ \hline
                    lozinka & CHAR & hash privremene lozinke \\ \hline
                    istice & DATE & datum isteka lozinke \\ \hline
                    
                    
                    
                \end{longtabu}

                Tablica \underbar{tajniBroj} sadrži popis tajnih brojeva za registraciju. Tablica tajniBroj je u \textit{Many-to-one} vezi s tablicom uloga (preko atributa sifUloga).
                \begin{longtabu} to \textwidth {|X[6, l]|X[6, l]|X[20, l]|}
                    
                    \hline \multicolumn{3}{|c|}{\textbf{tajniBroj}}     \\[3pt] \hline
                    \endfirsthead
                    
                    \hline \multicolumn{3}{|c|}{\textbf{tajniBroj}}     \\[3pt] \hline
                    \endhead
                    
                    \hline 
                    \endlastfoot

                    \textbf{broj} & INT & tajni broj \\ \hline
                    \cellcolor{LightBlue} sifUloga & INT & šifra uloge koju će korisnik poprimiti \\ \hline
                    idKorisnika & INT & neki identifikator osobe koja će iskoristiti tajni broj \\ \hline
                    
                    
                    
                \end{longtabu}

                Tablica \underbar{trgovina} sadrži informacije o trgovinama. Tablica trgovina je u \textit{One-to-many} vezi s tablicom artiklUTrgovini (preko atributa sifTrgovina). Tablica trgovina je u \textit{Many-to-one} vezi s tablicom korisnik (preko atributa email).
                \begin{longtabu} to \textwidth {|X[6, l]|X[6, l]|X[20, l]|}
                    
                    \hline \multicolumn{3}{|c|}{\textbf{trgovina}}     \\[3pt] \hline
                    \endfirsthead
                    
                    \hline \multicolumn{3}{|c|}{\textbf{trgovina}}     \\[3pt] \hline
                    \endhead
                    
                    \hline 
                    \endlastfoot

                    \textbf{sifTrgovina} & CHAR & šiftra trgovine \\ \hline
                    \cellcolor{LightBlue} email & VARCHAR & email vlasnika trgovine \\ \hline
                    lat & DECIMAL & geografska širina trgovine \\ \hline
                    lon & DECIMAL & geografska dužina trgovine \\ \hline
                    nazivTrgovina & NCHAR & naziv trgovine \\ \hline
                    adresa & VARCHAR & adresa trgovine \\ \hline
                    radnoVrijemePoc & TIME & početak radnog vremena trgovine \\ \hline
                    radnoVrijemeKraj & TIME & kraj radnog vremena \\ \hline
                    
                    
                    
                \end{longtabu}

                Tablica \underbar{uloga} sadrži popis uloga koje prijavljeni korisnik može poprimiti (administrator, trgovac ili kupac). Tablica uloga je u \textit{One-to-many} vezi s tablicom korisnik (preko atributa sifUloga) i tablicom tajniBroj (preko atributa sifUloga).
                \begin{longtabu} to \textwidth {|X[6, l]|X[6, l]|X[20, l]|}
                    
                    \hline \multicolumn{3}{|c|}{\textbf{uloga}}     \\[3pt] \hline
                    \endfirsthead
                    
                    \hline \multicolumn{3}{|c|}{\textbf{uloga}}     \\[3pt] \hline
                    \endhead
                    
                    \hline 
                    \endlastfoot

                    \textbf{sifUloga} & INT & šifra uloge korisnika \\ \hline
                    nazUloga & VARCHAR & naziv uloge prijavljenog korisnika \\ \hline
                    
                    
                    
                \end{longtabu}

                Tablica \underbar{vrsta} sadrži popis vrsta artikala. Tablica vrsta je u \textit{One-to-many} vezi s tablicom opisArtikla (preko atributa sifKategorija, sifPotkategorija i sifVrsta). Tablica vrsta je u \textit{Many-to-one} vezi s tablicom potkategorija (preko atributa sifKategorija i sifPotkategorija).
                \begin{longtabu} to \textwidth {|X[6, l]|X[6, l]|X[20, l]|}
                    
                    \hline \multicolumn{3}{|c|}{\textbf{vrsta}}     \\[3pt] \hline
                    \endfirsthead
                    
                    \hline \multicolumn{3}{|c|}{\textbf{vrsta}}     \\[3pt] \hline
                    \endhead
                    
                    \hline 
                    \endlastfoot

                    \cellcolor{LightBlue} \textbf{sifKategorija} & INT & šifra kategorije kojoj vrsta artikla pripada \\ \hline
                    \cellcolor{LightBlue} \textbf{sifPotkategorija} & INT & šifra potkategorije kojoj vrsta artikla pripada \\ \hline
                    \textbf{sifVrsta} & INT & šifra vrste \\ \hline
                    nazivVrsta & NCHAR VARYING & naziv vrste artikla \\ \hline
                    
                    
                    
                \end{longtabu}

                Tablica \underbar{zemlja} sadrži popis zemalja koje korisnik može odabrati kao zemlju podrijetla. Tablica zemlja je u \textit{One-to-many} vezi s tablicom opisArtikla (preko atributa sifZemlja).
                \begin{longtabu} to \textwidth {|X[6, l]|X[6, l]|X[20, l]|}
                    
                    \hline \multicolumn{3}{|c|}{\textbf{zemlja}}     \\[3pt] \hline
                    \endfirsthead
                    
                    \hline \multicolumn{3}{|c|}{\textbf{zemlja}}     \\[3pt] \hline
                    \endhead
                    
                    \hline 
                    \endlastfoot

                    \textbf{sifZemlja} & CHAR & šifra zemlje podrijetla \\ \hline
                    nazivZemlja & NCHAR VARYING & naziv zemlje podrijetla \\ \hline
                    
                    
                    
                \end{longtabu}

		%------------------------------------------------------------------------------
			\subsection{Dijagrami baze podataka}
		\begin{figure}[H]
		    \centering
			\includegraphics[scale=0.4]{dijagrami/db_uredaj.png}
			\caption{Baza podataka na uređaju}
			\label{fig:db_uredaj}
		\end{figure}
		
		\begin{figure}[H]
		    \centering
			\includegraphics[width=.9\linewidth]{dijagrami/db_posluzitelj.png}
			\caption{Baza podataka na poslužitelju}
			\label{fig:db_posluzitelj}
		\end{figure}
			
			\eject
			
			
		\section{Dijagrami razreda}
		
		Dijagram razreda na uređaju sadrži klase koje nisu dio generične funkcionalnosti, ali su ubačene radi preglednosti. To su List, Listitem, RoomDB, Favourites i About
		
		\begin{figure}[H]
		    \centering
			\includegraphics[width=1.0\linewidth]{dijagrami/class_android.png}
			\caption{Dijagram razreda na uređaju}
			\label{fig:class_android}
		\end{figure}
		
	
		\begin{figure}[H]
			\centering
			\includegraphics[width=1.0\linewidth]{dijagrami/class.png}
			\caption{Dijagram razreda na poslužitelju}
			\label{fig:class}
		\end{figure}
	
		\section{Dijagram stanja}
			Priložena su dva dijagrama stanja. Jedan prikazuje web stranicu na kojoj korisnik nije ulogiran, a drugi dijagram stanja prikazuje web stranicu na kojoj je ulogiran trgovac.
			
			Dijagram stanja koji opisuje web stranicu na kojoj korisnik nije ulogiran nudi nekoliko opcija: prijava, registracija kao kupac i registracija kao trgovac. Registrirati se kao može bilo tko samo jednom s istom email adresom. Za registraciju kao kupac nisu potrebni nikakvi preduvjeti, dok je za registraciju kao trgovac potreban odgovarajući kod.
			
			
			
			\begin{figure}[H]
				\centering
				\includegraphics[width=1.0\linewidth]{dijagrami/web-neulogiran.jpg}
				\caption{Dijagram stanja web stranice za neulogiranog korisnika}
				\label{fig:state-web-neulogiran}
			\end{figure}
		
			Sljedeći dijagram stanja opisuje web stranicu na kojoj je ulogiran trgovac. On ima već nabrojane opcije kao što su pregled trgovina, dodavanje trgovine, pregled proizvoda u trgovini, brisanje proizvoda u trgovini itd. Preduvjet za svaku akciju brisanja je naravno da proizvod koji se želi obrisati postoji. Također, što se tiče pregleda trgovine, trgovac može pregledati samo onu trgovinu za koju je on zadužen.
		
			\begin{figure}[H]
				\centering
				\includegraphics[width=1.0\linewidth]{dijagrami/web-trgovac.jpg}
				\caption{Dijagram stanja web stranice za trgovca}
				\label{fig:state-web-trgovac}
			\end{figure}
			
		
		\section{Dijagram aktivnosti}
			Na slici prikazan je dijagram aktivnosti. On obuhvaća sve aktivnosti koje se događaju na Android aplikaciji, serveru, serverskoj i Android bazi podataka. Obuhvaćeni skup aktivnosti obuhvaća sve od ulogiravanja korisnika na Android uređaju pa do dodavanja artikla na popis.
			
			Korisnik se dakle prvo ulogira, zatim odabere trgovinu te proizvod koji je ponuđen u toj trgovini. Odabirom proizvoda, prikazuje se detaljni opis tog proizvoda, njegova slika te opcije "upvote", "downvote" te "Dodaj na popis". Ovdje sam prikazao postupak dodavanja na popis. Jedan od preduvjeta je taj da postoji barem jedan popis na uređaju.
			
			\begin{figure}[H]
				\centering
				\includegraphics[width=1.0\linewidth]{dijagrami/dodavanje-art.png}
				\caption{Dijagram aktivnosti za prijavu i dodavanje artikla na popis (Android)}
				\label{fig:act-dodavanje-art}
			\end{figure}
		
		\section{Dijagram komponenti}
		
			Dijagram prikazuje komponente aplikacije koje komuniciraju preko sučelja. Web aplikacija od servera, uz podatke u JSON formatu, mora dobiti i izgled stranice - HTML i CSS. Android aplikacija ne traži HTML i CSS zbog toga što je prikaz primljenih podataka od servera (koji se također dobijaju u JSON formatu) definiran na samom Androidu.
			
			Server ima vlastitu bazu podataka s kojom komunicira preko sučelja koje dolazi u paketu Django, a svaki Android uređaj komunicira sa svojom bazom podataka preko sučelja definiranim paketom DBRoom.
			
			Također, Google pruža svoje sučelje koje se može iskoristiti za automatsko slanje mailova. Tu funkcionalnost koristi server kako bi korisniku poslao potvrdu o promjeni lozinke.
		
			\begin{figure}[H]
				\centering
				\includegraphics[width=1.0\linewidth]{dijagrami/component.png}
				\caption{Dijagram komponenti}
				\label{fig:cmp-component}
			\end{figure}
			
			
			\eject

